\documentclass{article}
\usepackage{geometry}
\usepackage{hyperref}

\geometry{
  top=1in,
  bottom=1.5in,
  left=1.25in,
  right=1.25in
}

\title{Rapport de projet IA41 - Rasende Roboter}
\author{SENGEL Noé, POURCINE Mattéo, FLEURET Gabriel}
\date{\today}


\begin{document}

\maketitle

\tableofcontents
\section{Mise en contexte}
Le but de ce projet est de mettre en application les différents algorithmes vu en cours pendant le semestre d'automne 2023 en IA41 (\href{https://fr.wikipedia.org/wiki/Algorithme_de_parcours_en_largeur}{Breadth-First Search}, \href
{	https://en.wikipedia.org/wiki/A*_search_algorithm}{A*}, \href{https://en.wikipedia.org/wiki/Depth-first_search}{Depth-first search}).
Ces algorithmes vont être appliqués sur un jeu de société allemand \href{https://fr.wikipedia.org/wiki/Ricochet_Robots}{Rasende Roboter} ou \href{https://fr.wikipedia.org/wiki/Ricochet_Robots}{Ricochet Robots} en français.\\ \\
Pour des raisons pratiques, nous avons choisis de travailler sur la première version du jeu sortie en 1999.

\end{document}